\documentclass[a4paper, 11pt]{article}

\usepackage[utf8]{inputenc}
\usepackage[T1, T2A]{fontenc}
\usepackage[english, russian]{babel}
\usepackage[top=2cm, bottom=2cm, left=2cm, right=2cm]{geometry}

\usepackage{pgfplots}
\pgfplotsset{compat=1.13}
\pgfplotsset{grid = major, grid style = {dashed}}

\usepackage{subcaption}
\usepackage{amsmath}

\usepackage{tabu}

\usepackage{pgfplotstable}
% recommended:
\usepackage{booktabs}
\usepackage{colortbl}

% pgfplotstable settings
\pgfplotstableset{
    every last row/.style = {after row = \hline},
    before row = {\hline},
    column type = {|c},
    every last column/.style={
        column type/.add={}{|}
    },
}

%Change label separator
\usepackage{caption}
\captionsetup[table]{labelformat=simple, labelsep = endash}
\captionsetup[figure]{labelformat=simple, labelsep = endash}

\RequirePackage{../../titlepage/TAYTitle}
\author{Овчаров Алексей}
\title{Исследование математической модели электромеханического объекта управления}
\labnumber{10}
\variant{3}
\begin{document}

\paragraph{Цель работы.} Изучение математических моделей и исследование характеристик электромеханического объекта управления, построенного на основе электродвигателя постоянного тока независимого возбуждения.

\paragraph{Исходные данные.} В таблице 1 представлены исходные данные для моделирIвания ДПТ.
\begin{table}[h!]
    \centering
    \caption{Исходные данные.}
    \begin{tabu}{|[1pt]c|c|c|c|c|c|c|c|c|c|[1pt]}
        \tabucline[1pt]{-}
        $U_\text{Н}$ & $n_0$ & $I_\text{Н}$ & $M_\text{Н}$ & $R$ & $U_\text{Я}$ & $J_\text{Д}$ & $T_\text{у}$ & $i_\text{р}$ & $J_\text{М}$\\
        В & об/мин & А & Н$\cdot$м & Ом & мс & кг$\cdot\text{м}^2$ & мс & &  кг$\cdot\text{м}^2$ \\ \hline
        36 & 4000 & 6.5 & 0.57 & 0.85 & 3 & $2.2\cdot10^{-4}$ & 6 & 40 & 0.15 \\
        \tabucline[1pt]{-}
    \end{tabu}
\end{table}

\section*{Рассчет параметров моделирования.}

В ходе эксперимента, изменяя нагрузочный момент, мы получили различные значения времени переходного процесса и установившиеся значения тока и угловой скорости, которые представлены в таблице ниже.

\begin{table}[h!]
    \caption{Данные о перехоных процессах при изменении момента нагрузки.}
    \centering
    \pgfplotstabletypeset[
        columns/t_p/.style = {column name = {$t_\text{п}$}},
        columns/w/.style = {column name = {$\omega_y$}},
        columns/I/.style = {column name = {$I_y$}},
        columns/M/.style = {column name = {$M_\text{СМ}$}},
    ]{../data/FullModel/Moment.dat}
\end{table}


\end{document}