\documentclass[a4paper, 11pt]{article}

\usepackage[utf8]{inputenc}
\usepackage[T1, T2A]{fontenc}
\usepackage[english, russian]{babel}

\usepackage{graphicx}
\usepackage[top=2cm, bottom=2cm, left=2cm, right=2cm]{geometry}

\usepackage{array}
\usepackage{tabu}

\usepackage{subcaption}
\usepackage{pgfplots}
\usetikzlibrary{pgfplots.polar}

\usepackage{pgfplotstable}
% recommended:
\usepackage{booktabs}
\usepackage{colortbl}

\usepackage{amsmath}
\usepackage{float}

\begin{document}

\paragraph{Цель работы.} Изучение частотных характеристик типовых динамических звеньев и способов их построения.
    
\section*{Исходные данные.}

\begin{center}
    \tabulinesep = 2mm
    \begin{tabu} spread 1em{|[1pt]l|c|[1pt]}
        \tabucline[1pt]{-}
        Тип звена & Передаточная функция \\ \hline
        Интегрирующее с замедлением & $\displaystyle\frac{K}{s(Ts + 1)}$ \\ \hline
        Изодромное & $\displaystyle\frac{K(Ts + 1)}{s}$ \\ \hline
        Консервативное & $\displaystyle\frac{K}{T^2s^2 + 1}$ \\
        \tabucline[1pt]{-}
    \end{tabu}
    \hspace{2cm}
    \begin{tabu}{|[1pt]c|c|c|[1pt]}
        \tabucline[1pt]{-}
        K & T & $\xi$ \\ \hline
        10 & 2 & 0.25 \\
        \tabucline[1pt]{-}
    \end{tabu}
\end{center}

\section*{Исследование интегрирующего звена с замедлением.}
Экспериментально построенные характеристики данного звена представлены ниже.
\begin{figure}[h!]
    \begin{subfigure}{0.5\textwidth}
        \centering
        \begin{tikzpicture}
            \begin{semilogxaxis} [
                    width = 0.9\textwidth,
                    xlabel = {$\omega$, 1/c},
                    ylabel = {$L_m$, дБ},
                ]
                \addplot table [x={w}, y={logA}] {data/data4.dat};
            \end{semilogxaxis}
        \end{tikzpicture}
        \caption{График ЛАЧХ.}
    \end{subfigure}
    \begin{subfigure}{0.5\textwidth}
        \centering
        \begin{tikzpicture}
            \begin{semilogxaxis} [
                    width = 0.9\textwidth,
                    xlabel = {$\omega$, 1/c},
                    ylabel = {$\psi$, градусы},
                ]
                \addplot table [x={w}, y={psi}] {data/data4.dat};
            \end{semilogxaxis}
        \end{tikzpicture}
        \caption{График ЛФЧХ.}
    \end{subfigure}
    
    \vspace{0.5cm}

    \begin{subfigure}{0.5\textwidth}
        \centering
        \begin{tikzpicture}
            \begin{polaraxis} [
                    width = 0.9\textwidth,
                    xlabel = {$A(\omega)$},
                    ylabel = {$\psi$, градусы},
                ]
                \addplot table [x={psi}, y={A}] {data/data4.dat};
            \end{polaraxis}
        \end{tikzpicture}
        \caption{График АФЧХ.}
    \end{subfigure}
    \begin{subfigure}{0.5\textwidth}
        \centering
        \begin{tikzpicture}
            \begin{polaraxis} [
                    width = 0.9\textwidth,
                    xlabel = {$L_m$, дБ},
                    ylabel = {$\psi$, градусы},
                ]
                \addplot table [x={psi}, y={logA}] {data/data4.dat};
            \end{polaraxis}
        \end{tikzpicture}
        \caption{График ЛАФЧХ.}
    \end{subfigure}
    \caption{Частотные характеристики интегрирующего звена с запаздыванием.}
\end{figure}
\newpage

\begin{minipage}[t]{0.6\textwidth}
    \vspace{-0.7cm}
    \begin{table}[H]
        \centering
        \caption{Таблица данных.}
        \pgfplotstabletypeset[
            columns/w/.style = {column name = {\boldmath$\omega$}},
            columns/logW/.style = {column name = {\boldmath$\lg{\omega}$}},
            columns/A/.style = {column name = {\boldmath$A(\omega)$}},
            columns/logA/.style = {column name = {\boldmath$20\lg{A(\omega)}$}},
            columns/psi/.style = {column name = {\boldmath$\psi$}},
            every head row/.style = {
            before row = \toprule, after row = \midrule
            },
            every last row/.style = {after row = \bottomrule},
        ]{data/data4.dat}
    \end{table}
\end{minipage}
\begin{minipage}[t]{0.4\textwidth}
    Передаточная функия исследуемого звена представлена в исходных данных. Из нее можно построить частотную функцию и найти выражения для АЧХ и ФЧХ.
    \begin{align}
        W(j\omega) = \frac{-K(T\omega + j)}{\omega(T^2\omega^2 + 1)} \\
        A(\omega) = \frac{K}{\omega\sqrt{T^2\omega^2 + 1}} \\
        \psi(\omega) = \arctg{\frac{1}{T\omega}}
    \end{align}
\end{minipage}

\section*{Исследование изодромного звена.}
\begin{minipage}[t]{0.6\textwidth}
    \vspace{-0.7cm}
    \begin{table}[H]
        \centering
        \caption{Таблица данных.}
        \pgfplotstabletypeset[
            columns/w/.style = {column name = {\boldmath$\omega$}},
            columns/logW/.style = {column name = {\boldmath$\lg{\omega}$}},
            columns/A/.style = {column name = {\boldmath$A(\omega)$}},
            columns/logA/.style = {column name = {\boldmath$20\lg{A(\omega)}$}},
            columns/psi/.style = {column name = {\boldmath$\psi$}},
            every head row/.style = {
            before row = \toprule, after row = \midrule
            },
            every last row/.style = {after row = \bottomrule},
        ]{data/data5.dat}
    \end{table}
\end{minipage}
\begin{minipage}[t]{0.4\textwidth}
    Передаточная функия исследуемого звена представлена в исходных данных. Из нее можно построить частотную функцию и найти выражения для АЧХ и ФЧХ.
    \begin{align}
        W(j\omega) = \frac{K(T\omega - j)}{\omega} \\
        A(\omega) = \frac{K\sqrt{T^2\omega^2 + 1}}{\omega} \\
        \psi(\omega) = -\arctg{\frac{1}{T\omega}}
    \end{align}
\end{minipage}
\begin{figure}[h!]
    \begin{subfigure}{0.5\textwidth}
        \centering
        \begin{tikzpicture}
            \begin{semilogxaxis} [
                    width = 0.9\textwidth,
                    xlabel = {$\omega$, 1/c},
                    ylabel = {$L_m$, дБ},
                ]
                \addplot table [x={w}, y={logA}] {data/data5.dat};
            \end{semilogxaxis}
        \end{tikzpicture}
        \caption{График ЛАЧХ.}
    \end{subfigure}
    \begin{subfigure}{0.5\textwidth}
        \centering
        \begin{tikzpicture}
            \begin{semilogxaxis} [
                    width = 0.9\textwidth,
                    xlabel = {$\omega$, 1/c},
                    ylabel = {$\psi$, градусы},
                ]
                \addplot table [x={w}, y={psi}] {data/data5.dat};
            \end{semilogxaxis}
        \end{tikzpicture}
        \caption{График ЛФЧХ.}
    \end{subfigure}
    
    \vspace{0.5cm}

    \begin{subfigure}{0.5\textwidth}
        \centering
        \begin{tikzpicture}
            \begin{polaraxis} [
                    width = 0.9\textwidth,
                    xlabel = {$A(\omega)$},
                    ylabel = {$\psi$, градусы},
                ]
                \addplot table [x={psi}, y={A}] {data/data5.dat};
            \end{polaraxis}
        \end{tikzpicture}
        \caption{График АФЧХ.}
    \end{subfigure}
    \begin{subfigure}{0.5\textwidth}
        \centering
        \begin{tikzpicture}
            \begin{polaraxis} [
                    width = 0.9\textwidth,
                    xlabel = {$L_m$, дБ},
                    ylabel = {$\psi$, градусы},
                ]
                \addplot table [x={psi}, y={logA}] {data/data5.dat};
            \end{polaraxis}
        \end{tikzpicture}
        \caption{График ЛАФЧХ.}
    \end{subfigure}
    \caption{Частотные характеристики изодромного звена.}
\end{figure}

\section*{Исследование консервативного звена.}

\begin{minipage}[t]{0.6\textwidth}
    \vspace{-0.7cm}
    \begin{table}[H]
        \centering
        \caption{Таблица данных.}
        \pgfplotstabletypeset[
            columns/w/.style = {column name = {\boldmath$\omega$}},
            columns/logW/.style = {column name = {\boldmath$\lg{\omega}$}},
            columns/A/.style = {column name = {\boldmath$A(\omega)$}},
            columns/logA/.style = {column name = {\boldmath$20\lg{A(\omega)}$}},
            columns/psi/.style = {column name = {\boldmath$\psi$}},
            every head row/.style = {
            before row = \toprule, after row = \midrule
            },
            every last row/.style = {after row = \bottomrule},
        ]{data/data7.dat}
    \end{table}
\end{minipage}
\begin{minipage}[t]{0.4\textwidth}
Ниже в таблице 3 представлены данные о частоте, фазе и амплитуде после графического сравнения графика входного воздействия и выходного. Из-за сильного искажения выходного сигнала (консерванивное звено обладает собственными колебаниями, амплитуда которых увеличивется с увеличением частоты) данные полностью снять не представлялось возможным. \par
Ниже представлены выражения частотных характеристик.
    \begin{align}
        W(j\omega) = \frac{K}{1 - T^2\omega^2} \\
        A(\omega) = \frac{K}{|1 - T^2\omega^2|}
    \end{align}
\end{minipage}
\newpage
\begin{equation}
    \psi(\omega) = \begin{cases}
        0, & \omega < \frac{1}{T} \\
        -180, & \omega > \frac{1}{T}
    \end{cases}
\end{equation}

Ниже на рисунке 3 представлены графики по данным, которые были сняты графически и полученные аналитически из выражения (8). Как видно из графиков, чем меньше частота колебаний - тем меньше амплитуда собственных колебаний системы.
\begin{figure}[h!]
    \begin{subfigure}{0.5\textwidth}
        \centering
        \begin{tikzpicture}
            \begin{semilogxaxis} [
                    width = 0.9\textwidth,
                    xlabel = {$\omega$, 1/c},
                    ylabel = {$L_m$, дБ},
                ]
                \addplot table [x={w}, y={logA}] {data/data7.dat};
                \addplot[red, thick, mark = none, style = dashed] table [x={w}, y={logA}] {data/data7-full.dat};
                \legend{\text{\scriptsize Эксперимент}, \text{\scriptsize Теория}}
            \end{semilogxaxis}
        \end{tikzpicture}
        \caption{График ЛАЧХ.}
    \end{subfigure}
    \begin{subfigure}{0.5\textwidth}
        \centering
        \begin{tikzpicture}
            \begin{semilogxaxis} [
                    width = 0.9\textwidth,
                    xlabel = {$\omega$, 1/c},
                    ylabel = {$\psi$, градусы},
                ]
                \addplot table [x={w}, y={psi}] {data/data7.dat};
                \addplot[red, thick, mark = none, style = dashed] table [x={w}, y={psi}] {data/data7-full.dat};
                \legend{\text{\scriptsize Эксперимент}, \text{\scriptsize Теория}}
            \end{semilogxaxis}
        \end{tikzpicture}
        \caption{График ЛФЧХ.}
    \end{subfigure}
    
    \vspace{0.5cm}

    \begin{subfigure}{0.5\textwidth}
        \centering
        \begin{tikzpicture}
            \begin{polaraxis} [
                    width = 0.9\textwidth,
                    xlabel = {$A(\omega)$},
                    ylabel = {$\psi$, градусы},
                ]
                \addplot table [x={psi}, y={A}] {data/data7.dat};
                \addplot[red, thick, mark = none, style = dashed] table [x={psi}, y={A}] {data/data7-full.dat};
                \legend{\text{\scriptsize Эксперимент}, \text{\scriptsize Теория}}
            \end{polaraxis}
        \end{tikzpicture}
        \caption{График АФЧХ.}
    \end{subfigure}
    \begin{subfigure}{0.5\textwidth}
        \centering
        \begin{tikzpicture}
            \begin{polaraxis} [
                    width = 0.9\textwidth,
                    xlabel = {$L_m$, дБ},
                    ylabel = {$\psi$, градусы},
                ]
                \addplot table [x={psi}, y={logA}] {data/data7.dat};
                \addplot[red, thick, mark = none, style = dashed] table [x={psi}, y={logA}] {data/data7-full.dat};
                \legend{\text{\scriptsize Эксперимент}, \text{\scriptsize Теория}}
            \end{polaraxis}
        \end{tikzpicture}
        \caption{График ЛАФЧХ.}
    \end{subfigure}
    \caption{Частотные характеристики консервативного звена.}
\end{figure}

\section*{Выводы}
В данной работе мы исследовали частотные характеристике трех звеньев: интегрирующего с запаздыванием, изодромного, консервативного. Получили экспериментально графики частотных характеристик и сравнили их с соответствующими выражениями. При отрицательных вещественных корнях характеристического уравнения, выходной гармонических сигнал будет колебаться с той же частотай, что и входной, но с измененной амплитудой и фазой. При чисто мнимых корнях проявляются собственные колебания системы, что приводит к генерации двух гармоник разной частоты (частоты входного сигнала и $1/T$). Соответственно выходной сигнал будет сильно изменен.

\end{document}
