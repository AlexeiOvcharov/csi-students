\documentclass[a4paper, 12pt]{article}

\usepackage[utf8]{inputenc}
\usepackage[T1, T2A]{fontenc}
\usepackage[english, russian]{babel}

\usepackage{graphicx}
\usepackage[top=2cm, bottom=2cm, left=2cm, right=2cm]{geometry}

\usepackage{array}
\usepackage{tabu}

\usepackage{subcaption}
\usepackage{pgfplots}
\usetikzlibrary{pgfplots.polar}
\pgfplotsset{compat=1.13}
\pgfplotsset{grid = major, grid style = {dashed}}

% PGFPlots Table ========================================================
\usepackage{pgfplotstable}
\renewcommand{\arraystretch}{1.5}
% recommended:
\usepackage{booktabs}
\usepackage{colortbl}
% pgfplotstable settings
\pgfplotstableset{
    columns/w/.style = {column name = {\boldmath$\omega$}, column type = |c},
    columns/logW/.style = {column name = {\boldmath$\lg{\omega}$}, column type = |c},
    columns/A/.style = {column name = {\boldmath$A(\omega)$}, column type = |c},
    columns/logA/.style = {column name = {\boldmath$20\lg{A(\omega)}$}, column type = |c},
    columns/psi/.style = {column name = {\boldmath$\psi$}, column type = |c|},
    every head row/.style = {before row = \hline},
    after row = {[1mm] \hline},
}

\usepackage{amsmath}
\usepackage{float}

% Table and figure setting ==============================================
\usepackage{threeparttable}
%Change label separator
\usepackage{caption}
\captionsetup[table]{labelformat=simple, labelsep = endash, justification = raggedright, singlelinecheck = off}
\captionsetup[figure]{labelformat=simple, labelsep = endash, name = Рисунок}

% Paragraph indent
\usepackage{indentfirst}

% Межстрочный интервал - 2


\usepackage{../titlepage/TAYTitle}
\author{Овчаров Алексей}
\title{Экспериментальное построение частотных характеристик типовых динамических звеньев}
\labnumber{9}
\variant{3}

\begin{document}
\maketitle

\section{Задание}
\textbf{Цель работы} - изучение частотных характеристик типовых динамических звеньев и способов их построения. \par
Если на вход устойчивого линейного звена с передаточной функцией $W(s)$ подается гармонический сигнал $g(t) = g_m\sin{\omega t}$, то на его выходе в установившемся режиме будет гармонический сигнал $y(t) = y_m\sin{(\omega t + \psi)}$. Задача состоим в том, чтобы определить зависимость амплитуды $y_m(\omega)$ и разности фаз между выходым сигналом и входным $\psi(\omega)$ от частоты входного сигнала. Полученные графики получили название: амплитудно-частотная характеристика (АЧХ) и фазо-частотная характеристика (ФЧХ). \par
В данной реботе необходимо получить АЧХ и ФЧХ линейных динамических звеньев, представленных в таблице 1, подставив в них параметры, указанные в таблице 2. После чего, на основе двух предыдущих характеристик, построить амплитудно-фазовую характеристику (АФЧХ).

\begin{table}[H]
    \tabulinesep = 2mm
    \caption{Исходные элементарные звенья}
    \begin{tabu} spread 1em{|[1pt]l|c|[1pt]}
        \tabucline[1pt]{-}
        Тип звена & Передаточная функция \\ \hline
        Интегрирующее с замедлением & $\displaystyle\frac{K}{s(Ts + 1)}$ \\ \hline
        Изодромное & $\displaystyle\frac{K(Ts + 1)}{s}$ \\ \hline
        Консервативное & $\displaystyle\frac{K}{T^2s^2 + 1}$ \\
        \tabucline[1pt]{-}
    \end{tabu}
\end{table}
\begin{table}[H]
    \tabulinesep = 2mm
    \caption{Параметры}
    \begin{tabu}{|[1pt]c|c|c|[1pt]}
        \tabucline[1pt]{-}
        K & T & $\xi$ \\ \hline
        10 & 2 & 0.25 \\
        \tabucline[1pt]{-}
    \end{tabu}
\end{table}

\newpage
\section{Исследование интегрирующего звена с замедлением}

В таблице 3 представлены данные, снятые по графикам переходных процессов. \par
\begin{table}[h!]
    \centering
    \begin{threeparttable}
        \caption{Полученные данные} \label{tab:perflogcross}
        \pgfplotstabletypeset[]{data/data4.dat}
    \end{threeparttable}
\end{table}
Передаточная функия исследуемого звена представлена в таблице 1. Из нее можно построить частотную функцию и найти выражения для АЧХ и ФЧХ.
\begin{align}
    W(j\omega) = \frac{-K(T\omega + j)}{\omega(T^2\omega^2 + 1)} \\
    A(\omega) = \frac{K}{\omega\sqrt{T^2\omega^2 + 1}} \\
    \psi(\omega) = \arctg{\frac{1}{T\omega}}
\end{align}

\newpage
Экспериментально построенные характеристики данного звена представлены ниже.
\begin{figure}[h!]
    \begin{subfigure}{0.5\textwidth}
        \centering
        \begin{tikzpicture}
            \begin{semilogxaxis} [
                    width = 0.9\textwidth,
                    xlabel = {$\omega$, 1/c},
                    ylabel = {$L_m$, дБ},
                ]
                \addplot table [x={w}, y={logA}] {data/data4.dat};
            \end{semilogxaxis}
        \end{tikzpicture}
        \caption{График ЛАЧХ}
    \end{subfigure}
    \begin{subfigure}{0.5\textwidth}
        \centering
        \begin{tikzpicture}
            \begin{semilogxaxis} [
                    width = 0.9\textwidth,
                    xlabel = {$\omega$, 1/c},
                    ylabel = {$\psi$, градусы},
                ]
                \addplot table [x={w}, y={psi}] {data/data4.dat};
            \end{semilogxaxis}
        \end{tikzpicture}
        \caption{График ЛФЧХ}
    \end{subfigure}
    
    \vspace{0.5cm}

    \begin{subfigure}{0.5\textwidth}
        \centering
        \begin{tikzpicture}
            \begin{polaraxis} [
                    width = 0.9\textwidth,
                    xlabel = {$A(\omega)$},
                    ylabel = {$\psi$, градусы},
                ]
                \addplot table [x={psi}, y={A}] {data/data4.dat};
            \end{polaraxis}
        \end{tikzpicture}
        \caption{График АФЧХ}
    \end{subfigure}
    \begin{subfigure}{0.5\textwidth}
        \centering
        \begin{tikzpicture}
            \begin{polaraxis} [
                    width = 0.9\textwidth,
                    xlabel = {$L_m$, дБ},
                    ylabel = {$\psi$, градусы},
                ]
                \addplot table [x={psi}, y={logA}] {data/data4.dat};
            \end{polaraxis}
        \end{tikzpicture}
        \caption{График ЛАФЧХ}
    \end{subfigure}
    \caption{Частотные характеристики интегрирующего звена с запаздыванием}
\end{figure}

\newpage
\section{Исследование изодромного звена}
В таблице 4 представлены данные, снятые по графикам переходных процессов.
\begin{table}[h!]
    \centering
    \begin{threeparttable}
        \caption{Полученные данные}
        \pgfplotstabletypeset[]{data/data5.dat}
    \end{threeparttable}
\end{table}

Передаточная функия исследуемого звена представлена в таблице 1. Из нее можно построить частотную функцию и найти выражения для АЧХ и ФЧХ.
\begin{align}
    W(j\omega) = \frac{K(T\omega - j)}{\omega} \\
    A(\omega) = \frac{K\sqrt{T^2\omega^2 + 1}}{\omega} \\
    \psi(\omega) = -\arctg{\frac{1}{T\omega}}
\end{align}

\newpage
\begin{figure}[h!]
    \begin{subfigure}{0.5\textwidth}
        \centering
        \begin{tikzpicture}
            \begin{semilogxaxis} [
                    width = 0.9\textwidth,
                    xlabel = {$\omega$, 1/c},
                    ylabel = {$L_m$, дБ},
                ]
                \addplot table [x={w}, y={logA}] {data/data5.dat};
            \end{semilogxaxis}
        \end{tikzpicture}
        \caption{График ЛАЧХ}
    \end{subfigure}
    \begin{subfigure}{0.5\textwidth}
        \centering
        \begin{tikzpicture}
            \begin{semilogxaxis} [
                    width = 0.9\textwidth,
                    xlabel = {$\omega$, 1/c},
                    ylabel = {$\psi$, градусы},
                ]
                \addplot table [x={w}, y={psi}] {data/data5.dat};
            \end{semilogxaxis}
        \end{tikzpicture}
        \caption{График ЛФЧХ}
    \end{subfigure}
    
    \vspace{0.5cm}

    \begin{subfigure}{0.5\textwidth}
        \centering
        \begin{tikzpicture}
            \begin{polaraxis} [
                    width = 0.9\textwidth,
                    xlabel = {$A(\omega)$},
                    ylabel = {$\psi$, градусы},
                ]
                \addplot table [x={psi}, y={A}] {data/data5.dat};
            \end{polaraxis}
        \end{tikzpicture}
        \caption{График АФЧХ}
    \end{subfigure}
    \begin{subfigure}{0.5\textwidth}
        \centering
        \begin{tikzpicture}
            \begin{polaraxis} [
                    width = 0.9\textwidth,
                    xlabel = {$L_m$, дБ},
                    ylabel = {$\psi$, градусы},
                ]
                \addplot table [x={psi}, y={logA}] {data/data5.dat};
            \end{polaraxis}
        \end{tikzpicture}
        \caption{График ЛАФЧХ}
    \end{subfigure}
    \caption{Частотные характеристики изодромного звена}
\end{figure}

\newpage
\section{Исследование консервативного звена}
В таблице 5 представлены данные о частоте, фазе и амплитуде после сравнения графиков входного и выходного сигналов. Из-за сильного искажения выходного сигнала (консерванивное звено обладает собственными колебаниями, амплитуда которых увеличивется с увеличением частоты) данные полностью снять не представлялось возможным. \par
\begin{table}[h!]
    \centering
    \begin{threeparttable}
        \caption{Полученные данные}
        \pgfplotstabletypeset[]{data/data7.dat}
    \end{threeparttable}
\end{table}

Ниже представлены выражения частотных характеристик.
\begin{align}
    W(j\omega) & = \frac{K}{1 - T^2\omega^2} \\
    A(\omega) & = \frac{K}{|1 - T^2\omega^2|} \\ 
    \psi(\omega) & = \begin{cases}
        0, & \omega < \frac{1}{T} \\
        -180, & \omega > \frac{1}{T}
    \end{cases}
\end{align}

\newpage
На рисунке 3 представлены графики по данным, которые были сняты графически и полученные аналитически из выражения (8). Как видно из графиков, чем меньше частота колебаний - тем меньше амплитуда собственных колебаний системы.
\begin{figure}[h!]
    \begin{subfigure}{0.5\textwidth}
        \centering
        \begin{tikzpicture}
            \begin{semilogxaxis} [
                    width = 0.9\textwidth,
                    xlabel = {$\omega$, 1/c},
                    ylabel = {$L_m$, дБ},
                ]
                \addplot table [x={w}, y={logA}] {data/data7.dat};
                \addplot[red, thick, mark = none, style = dashed] table [x={w}, y={logA}] {data/data7-full.dat};
                \legend{\text{\scriptsize Моделирование}, \text{\scriptsize Теория}}
            \end{semilogxaxis}
        \end{tikzpicture}
        \caption{График ЛАЧХ}
    \end{subfigure}
    \begin{subfigure}{0.5\textwidth}
        \centering
        \begin{tikzpicture}
            \begin{semilogxaxis} [
                    width = 0.9\textwidth,
                    xlabel = {$\omega$, 1/c},
                    ylabel = {$\psi$, градусы},
                ]
                \addplot table [x={w}, y={psi}] {data/data7.dat};
                \addplot[red, thick, mark = none, style = dashed] table [x={w}, y={psi}] {data/data7-full.dat};
                \legend{\text{\scriptsize Моделирование}, \text{\scriptsize Теория}}
            \end{semilogxaxis}
        \end{tikzpicture}
        \caption{График ЛФЧХ}
    \end{subfigure}
    
    \vspace{0.5cm}

    \begin{subfigure}{0.5\textwidth}
        \centering
        \begin{tikzpicture}
            \begin{polaraxis} [
                    width = 0.9\textwidth,
                    xlabel = {$A(\omega)$},
                    ylabel = {$\psi$, градусы},
                ]
                \addplot table [x={psi}, y={A}] {data/data7.dat};
                \addplot[red, thick, mark = none, style = dashed] table [x={psi}, y={A}] {data/data7-full.dat};
                \legend{\text{\scriptsize Моделирование}, \text{\scriptsize Теория}}
            \end{polaraxis}
        \end{tikzpicture}
        \caption{График АФЧХ}
    \end{subfigure}
    \begin{subfigure}{0.5\textwidth}
        \centering
        \begin{tikzpicture}
            \begin{polaraxis} [
                    width = 0.9\textwidth,
                    xlabel = {$L_m$, дБ},
                    ylabel = {$\psi$, градусы},
                ]
                \addplot table [x={psi}, y={logA}] {data/data7.dat};
                \addplot[red, thick, mark = none, style = dashed] table [x={psi}, y={logA}] {data/data7-full.dat};
                \legend{\text{\scriptsize Моделирование}, \text{\scriptsize Теория}}
            \end{polaraxis}
        \end{tikzpicture}
        \caption{График ЛАФЧХ}
    \end{subfigure}
    \caption{Частотные характеристики консервативного звена}
\end{figure}

\newpage
\section*{Выводы}
В данной работе мы исследовали частотные характеристике трех звеньев: интегрирующего с запаздыванием, изодромного, консервативного. Получили экспериментально графики частотных характеристик и сравнили их с соответствующими выражениями. \par
При отрицательных вещественных корнях характеристического уравнения, выходной гармонических сигнал будет колебаться с той же частотай, что и входной, но с измененной амплитудой и фазой. При чисто мнимых корнях проявляются собственные колебания системы, что приводит к генерации двух гармоник разной частоты (частоты входного сигнала и $1/T$). Соответственно выходной сигнал будет сильно изменен. \par
Как видно из рисунка 1, 2 и 3, при частоте $\omega_c = 1/T = 0.5$ ЛАЧХ изменяет наклон, что соответствует теории. Фазовый сдвиг интегрирующего звена с замедление изменяется с $-90^{\circ}$ до $-180^{\circ}$, изодромного соответсвенно от $-90^\circ$ до $0^\circ$ и у консервативного от $0^\circ$ до $-180^\circ$.


\end{document}
