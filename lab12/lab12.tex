\documentclass[a4paper, 12pt]{article}

\usepackage[utf8]{inputenc}
\usepackage[T1, T2A]{fontenc}
\usepackage[english, russian]{babel}
\usepackage[top=2cm, bottom=2cm, left=2cm, right=2cm]{geometry}

\usepackage{pgfplots}
\usepgfplotslibrary{polar}
\pgfplotsset{compat=1.13}
\pgfplotsset{grid = major, grid style = {dashed}}

\usepackage{subcaption}
\usepackage{amsmath}

\usepackage{tabu}

\usepackage{float}

% PGFPlots Table ========================================================
\usepackage{pgfplotstable}
\renewcommand{\arraystretch}{1.3}
% recommended:
\usepackage{booktabs}
\usepackage{colortbl}
% pgfplotstable settings
\pgfplotstableset{
    every head row/.style = {before row = \hline},
    after row = {[1mm] \hline},
    column type = {|c},
    every last column/.style={
        column type/.add={}{|},
    },   
}
\usepackage{threeparttable}
\renewcommand{\TPTminimum}{0.6\linewidth}

% Paragraph indent
\usepackage{indentfirst}
\setlength{\parindent}{15mm}

%Change label separator
\usepackage{caption}
\captionsetup[table]{labelformat=simple, labelsep = endash, justification = raggedright, singlelinecheck = off, width = 0.75\textwidth}
\captionsetup[figure]{labelformat=simple, labelsep = endash, name = Рисунок}

\usepackage{../titlepage/TAYTitle}
\author{Овчаров Алексей}
\title{Анализ линейных непрерывных систем с использованием прикладного пакета matlab control system 
toolbox.}
\labnumber{12}
\variant{3}

\begin{document}

\maketitle

\begin{center}
\section{Задача}
\end{center} \par
\textbf{Целью работы} является исследование динамических и частотных характеристик, анализ структурных свойств и устойчивости линейных непрерывных систем с помощью прикладного пакета matlab. \par
В качестве объекта исследования выбраны линейные непрерывные динамические стационарные системы. Исходная модель разомкнутой системы представляется в форме вход-выход и описывается передаточной функцией вида: 
\begin{equation} 
    W(s) = \frac{b_1s + b_0}{s(a_2s^2 + a_1s + a_0)}
\end{equation} \par
Значения коэффициентов $a_0, a_1, a_2, b_0, b_1$ представлены в таблице 1. \par
\begin{table} [h!]
    \centering
    \begin{threeparttable}
        \caption{Коэффициенты передаточной функции}
        \begin{tabular}{|c|c|c|c|c|}
            \hline
            $a_0$ & $a_1$ & $a_2$ & $b_0$ & $b_1$ \\ \hline
            1 & 2 & 3 & 2 & 1 \\ \hline
        \end{tabular}
    \end{threeparttable}
\end{table}
Далее необходимо перейти от исходной разомкнутой системы к замкнутой системе с жесткой отрицательной обратной связью и провести ее анализ в соответствии с порядком выполнения работы.

\newpage
\begin{center}
\section{Анализ разомкнутой системы}
\end{center} \par
Передаточная функция разомкнутой системы представлена ниже:
\begin{equation}
    W(s) = \frac{s + 2}{3s^3 + 2s^2 + s}
\end{equation}

\noindent
\begin{minipage}[t]{0.5\textwidth}
    \begin{figure} [H]
        \centering
        \begin{tikzpicture}
            \begin{axis} [
                width = 0.9\textwidth,
                xlabel = {Re},
                ylabel = {Im},
                grid = major,
                grid style = {dashed},
                legend pos = north west,
            ]
                \addplot[only marks, mark = x, mark size = 5pt] coordinates {(-0.3333, 0.4714) (-0.3333, -0.4714) (0, 0)}; 
                \addplot[only marks, mark = o, mark size = 5pt] coordinates {(-2, 0)};
                \legend{полюса, нули};
            \end{axis}
        \end{tikzpicture}
        \caption{Нули и полюса}
    \end{figure}
\end{minipage}
\begin{minipage}[t]{0.5\textwidth}
    \vspace{0.5cm}
    Из исходной системы можем найти нули и полюса.
    \begin{align*}
        p_1 & = -2 \\
        z_1 & = 0 & z_{2, 3} = -\frac{1}{3} \pm i\frac{\sqrt{2}}{3}
    \end{align*}
    где p - полюса, z - нули. Графическое изображение найденых решений представлено на рисунке 1.
\end{minipage}
\vspace{0.5cm} \par

Далее построим логарифмические амплитудночастотные и фазочастотные характеристики. Они представлены на рисунке 2.

\begin{figure}[h!]
\begin{subfigure}{\textwidth}
    \centering
    \begin{tikzpicture}
        \begin{semilogxaxis} [
            width = 0.9\textwidth,
            height = 6cm,
            ylabel = {$20L(\omega)$, дБ},
            xlabel = {$\omega$, 1/c}, grid = major,
            grid style = {dashed},
            xmin = 10e-3, xmax = 10e+1,
            extra y ticks = {0},
            extra x ticks = {0.919},
            extra x tick labels = {0.92},
            xtick = {10e-3, 10e-2, 10e+0, 10e+1},
            ytick = {50, -50, -100},
        ]
            \addplot[red, mark = none, thick, smooth, solid] table [x = w, y = lgA] {data/bode.csv};
            \draw[dashed] (0.001, 0) -- (0.919, 0);
            \draw[dashed] (0.919, 0) -- (0.919, -101);
            \draw[dashed] (0.707, -101) -- (0.707, 0);
            \draw (0.707, 0) -- (0.707, 6.02);
            \draw[fill] (0.919, 0) circle (0.05cm) node[anchor = south west] {$\omega_c$};
        \end{semilogxaxis}
    \end{tikzpicture}
\end{subfigure}

\begin{subfigure}{\textwidth}
    \centering
    \begin{tikzpicture}
        \begin{semilogxaxis} [
            width = 0.9\textwidth,
            height = 6cm,
            ylabel = {$\varphi$, град.},
            xlabel = {$\omega$, 1/c}, grid = major,
            grid style = {dashed},
            xmin = 10e-3, xmax = 10e+1,
            ytick = {-100, -150, -180, -200},
            xtick = {10e-3, 10e-2, 10e+0, 10e+1},
            extra x ticks = {0.919, 0.707},
            extra x tick labels = {, 0.72},
        ]
            \addplot[blue, mark = none, thick, smooth, solid] table [x = w, y = phi] {data/bode.csv};
            \draw[dashed] (0.001, -180) -- (0.919, -180);
            \draw[dashed] (0.919, -180) -- (0.919, -70);
            \draw[dashed] (0.707, -180) -- (0.707, -70);
            \draw (0.919, -180) -- (0.919, -195.2);
        \end{semilogxaxis}
    \end{tikzpicture}
\end{subfigure}
\caption{ЛАЧХ и ЛФЧХ}
\end{figure}

\newpage
\hspace*{-\parindent}%
\begin{minipage}[t]{0.4\textwidth}
    \vspace{-1cm}
    \begin{figure}[H]
    \centering
        \begin{tikzpicture}
            \begin{axis}[
                grid = major,
                grid style = {dashed},
                width = 0.9\textwidth,
                height = 2.08\textwidth,
                ymin = -10, ymax = 1,
                xmin = -4, xmax = 0,
                xlabel = {Re}, ylabel = {Im},
            ]
                \addplot[blue, mark = none, thick, smooth, solid] table {data/nyquist.csv};
                \draw[dashed] (0, 0) circle (1);
                \draw[fill] (-1, 0) circle (0.05cm);
            \end{axis}
        \end{tikzpicture}
    \caption{АФЧХ}
    \end{figure}
\end{minipage}
\begin{minipage}[t]{0.6\textwidth}
    \parindent = 15mm
    \parПо ЛАЧХ и ЛФЧХ можно найти частоту среза $\omega_c$, запас устойчивости по амплитуде $A_\text{з}$ и по фазе $\varphi_\text{з}$.
    \begin{align*}
        \omega_c & = 0.919 & A_\text{з} & = -2 & \varphi_\text{з} = -15.2^\circ
    \end{align*}
    Все эти точки отмечены соотвественно на рисунке 2. Далее на основании полученых характеристик можем постороить амплитудночастотую характеристику. Она изображена на рисунке 3. \par
    Для определения устойчивости воспользуетмся критерием Найквиста. Специально для удобства определения была начерчена единичная окружность. Как видно из рисунка 3 системе не устойчива по данному критерию, поскольку график огибает точку $(-1, 0)$.
\end{minipage}

\newpage
\begin{center}
\section{Анализ замкнутой системы}
\end{center} \par

Передаточная функция с коэффициентом обратной связи $K$ записана ниже.
\begin{equation}
    W_\text{замк.}(s) = \frac{s + 2}{3s^3 + 2s^2 + (1 + K)s + 2K}
\end{equation} \par
Далее на рисунке 4 представлен графики корней при разных коэффициентах обратой связи $K \in [0, 108]$. \par
\begin{figure} [h!]
    \centering
    \begin{tikzpicture}
        \begin{axis} [
            width = 0.9\textwidth,
            height = 0.5\textwidth,
            xlabel = {Re},
            ylabel = {Im},
            grid = major,
            grid style = {dashed},
            legend pos = north west,
            xmin = -2.1, xmax = 0.8,
        ]
            \addplot[mark = none, blue, mark size = 3pt] table [x = Re1, y = Im1] {/home/senserlex/homeworks/TAY/csi-students/lab12/data/rlocus.csv};
            \addplot[mark = none, blue] table [x = Re2, y = Im2] {/home/senserlex/homeworks/TAY/csi-students/lab12/data/rlocus.csv};
            \addplot[mark = none, blue] table [x = Re3, y = Im3] {/home/senserlex/homeworks/TAY/csi-students/lab12/data/rlocus.csv};
            \addplot[only marks, mark = o, mark size = 5pt] coordinates {(-2, 0)};
            \draw[fill] (0, 0) circle (0.07cm);
            \draw[fill] (-0.33333, 0.4714) circle (0.07cm);
            \draw[fill] (-0.33333, -0.4714) circle (0.07cm);
            \draw (0, 0) -- (-0.2, 2);
            \draw[] (-0.33333, 0.4714) -- (-0.2, 2);
            \draw (-0.33333, -0.4714) -- (-0.2, 2);
            \draw (-0.2, 2) node[anchor = south] {K = 0};
            \legend{полюса, , , нули};
        \end{axis}
    \end{tikzpicture}
    \caption{Нули и полюса}
\end{figure}
Как видно, при $K = 0$ система имеет 1 нулевой корень и находится на границе устойчивости нейтрального типа, при увеличении $K$, система становится устойчивой, при этом вещественная часть комплексно сопряженных корней стемится в правую полуплоскость, что ведет к неустойчивости системы. \par
Пользуясь критерием Гурвица можно вывести, что система будет устойчива при следующих $K$:
\begin{equation}
    0 > K > 0.5
\end{equation}
Это также соответствует рисунку 4. Выберем $K = 0.15$ и будем дальше с ним работать.\par

\noindent
\begin{minipage}[t]{0.5\textwidth}
    \begin{figure} [H]
        \centering
        \begin{tikzpicture}
            \begin{axis} [
                width = 0.8\textwidth,
                xlabel = {Re},
                ylabel = {Im},
                grid = major,
                grid style = {dashed},
                legend pos = north west,
                xmin = -2.1, xmax = 0,
            ]
                \addplot[only marks, mark = x, mark size = 5pt] coordinates {(-0.1504, 0.5007) (-0.1504, -0.5007) (-0.3659, 0)}; 
                \addplot[only marks, mark = o, mark size = 5pt] coordinates {(-2, 0)};
                \legend{полюса, нули};
            \end{axis}
        \end{tikzpicture}
        \caption{Нули и полюса}
    \end{figure}
\end{minipage}
\begin{minipage}[t]{0.5\textwidth}
    \vspace{0.5cm}
    \parindent = 15mm
    В таком случае набор корней будет следующим (также корни изображены на рисунке 5):
    \begin{align*}
        p_1 & = -2 & 
        z_1 & =  -0.3659 \\
        z_{2, 3} & =  -0.1504 \pm 0.5007i &
    \end{align*} \par
    Как видно из рисунка 5 система устойчива. Степень устойчивости системы равна $Re(z_2) = Re(z_3) = -0.1504$.
\end{minipage}

\newpage
\parНа рисунке 6 и 7 представлены графики переходной и весовой функций замкнутой системы.

\begin{figure} [h!]
    \centering
    \begin{tikzpicture}
        \begin{axis} [
            width = \textwidth,
            height = 7cm,
            xlabel = {t, c},
            ylabel = {y},
            grid = major,
            grid style = {dashed},
            xmin = 0, xmax = 37.504, ymin = 0,
            extra y ticks = {6.66, 6.993, 6.327, 7.6913},
            extra y tick labels = {6.66, , , 7.69},
            extra x ticks = {8.306, 16.17},
            extra y tick style = {grid style = {black, dashed}},
            extra tick style={% changes for all extra ticks
                tick align=outside,
                tick pos=left,
                grid style={dotted,black}
            },
            ytick = {0, 2, 4},
            xtick = {0, 5, 10, 15, 20, 25, 30, 35},
        ]
            \addplot[blue, mark = none] table {data/step.csv};
            \draw[fill] (8.3062, 7.6913) circle (0.07cm);
            \draw[fill] (16.1701, 6.327) circle (0.07cm);
        \end{axis}
    \end{tikzpicture}
    \caption{Переходная функция}
\end{figure}
\begin{figure} [h!]
    \centering
    \begin{tikzpicture}
        \begin{axis} [
            width = \textwidth,
            height = 7cm,
            xlabel = {t, c},
            ylabel = {y},
            grid = major,
            grid style = {dashed},
            xmin = 0, xmax = 37.504,
            xtick = {0, 5, 10, 15, 20, 25, 30, 35},
        ]
            \addplot[blue, mark = none] table {data/impulse.csv};
        \end{axis}
    \end{tikzpicture}
    \caption{Весовая функция} 
\end{figure}
Время переходного процесса $t_\text{п}$ и перерегулирование $\sigma$ угазаны ниже:
\begin{align*}
    t_\text{п} & = 16.17 c & \sigma & = 15.4\%
\end{align*} \par
Далее приведем модель 3 к модели ВСВ. Она выглядит следующим образом:
\begin{equation}
    \begin{cases}
        \dot{X} = \begin{bmatrix}
            0 & 1 & 0 \\
            0 & 0 & 1 \\
            0 & -\frac{1}{3} & -\frac{2}{3}
        \end{bmatrix}X + \begin{bmatrix}
            0 \\ 0 \\ 1
        \end{bmatrix}U \\
        y = \begin{bmatrix}\frac{2}{3} & \frac{1}{3} & 0\end{bmatrix} X
    \end{cases}
\end{equation}
где $x = \begin{bmatrix} x_1 & x_2 & x_3 \end{bmatrix}^T$.
\newpage
\par
Теперь можно составить матрицы управляемости $W_y$ и наблюдаемости $W_\text{н}$ для опредления управляемости и наблюдаемости модели.

\begin{align*}
    W_y & = \begin{bmatrix}
        0 & 0 & 1 \\
        0 & 1 & -\frac{2}{3} \\
        0 & -\frac{2}{3} & \frac{2}{9} \\
    \end{bmatrix} & 
    W_\text{н} & = \begin{bmatrix}
        \frac{2}{3} & 0 & 1 \\
        \frac{1}{3} & \frac{2}{3} & -\frac{1}{9} \\
        0 & \frac{1}{3} & -\frac{4}{9} \\
    \end{bmatrix}
\end{align*}
Поскольку $rang\{W_y\} = 2$ система является не управляемой. При этом система является полностью наблюдаемой, поскольку $rang\{W_\text{н}\} = 3$.

\newpage
\begin{center}
\section*{Выводы}
\end{center} \par
В данной работы мы ислледовали разомкнутую систему. Получили ее корни и сравнили коренной критерий и Найквиста. Неустойчивая система по Накйквисту является системой на нейтральной границе устойчивости по корневому критерию. Также мы постоили различные частотыне характеристики, и по ним определили запас устойчивости по амплитуде и фазе. \par
Также осуществили анализ замкнутой системы по размокнутой(рисунок 4). При К = 0 система становится разомкнутой. При увеличении К до 1/2 система будет устойчивой. \par
Далее мы построили переходную и весовую функцию, по которым определили время переходного процесса и перерегулирование, после чего преобразовали модель вход-выход замкнутой системы в каноническую управляемой форму модели ВСВ. \par
По полученной модели ВСВ мы проверили ее на управляемость и наблюдаемость. Система оказалась не полностью управляемой, поскольку элемент $a_{31}$ матрицы состояний равен нулю.
\end{document}